%%%%%%%%%%%%%%%%%%%%%%%%%%%%%%%%%%%%%%%%%%%%%%%%%%%%%%%%%%%%%%%%
%
%  Template for homework of DSP@ustc.
%
%  Fill in your name, student id, homework Number
%  
%   Please compile with XelaTex
%%%%%%%%%%%%%%%%%%%%%%%%%%%%%%%%%%%%%%%%%%%%%%%%%%%%%%%%%%%%%%%%


\documentclass[11pt,letter,notitlepage,UTF8]{ctexart}
%Mise en page
\usepackage{listings}
\usepackage{xcolor}
\lstset{
	numbers=left, 
	numberstyle= \tiny, 
	keywordstyle= \color{ blue!70},
	commentstyle= \color{red!50!green!50!blue!50}, 
	frame=shadowbox, % 阴影效果
	rulesepcolor= \color{ red!20!green!20!blue!20} ,
	escapeinside=``, % 英文分号中可写入中文
	xleftmargin=2em,xrightmargin=2em, aboveskip=1em,
	framexleftmargin=2em
}

\usepackage[left=2.5cm, right=2.5cm, lines=45, top=1.5in, bottom=0.7in]{geometry}
\usepackage{fancyhdr}
\usepackage{fancybox}
\usepackage{graphicx}
\usepackage{pdfpages} 
\usepackage{enumitem}
\usepackage{algorithm}
\usepackage{algorithmic}
\newcommand\Loadedframemethod{TikZ}
\usepackage[framemethod=\Loadedframemethod]{mdframed}

\usepackage{amssymb,amsmath}
\usepackage{amsthm}
\usepackage{thmtools}
\newtheorem{lemma}{Lemma}

\usepackage{subfigure}
\usepackage{threeparttable}

\pagestyle{fancy}
%%%%%%%%%%%%%%%%%%%%%%%%
%% Define the Exercise environment %%
%%%%%%%%%%%%%%%%%%%%%%%%
\mdtheorem[
topline=false,
rightline=false,
leftline=false,
bottomline=false,
leftmargin=-10,
rightmargin=-10
]{exercise}{\textbf{习题}}
%%%%%%%%%%%%%%%%%%%%%%%
%% End of the Exercise environment %%
%%%%%%%%%%%%%%%%%%%%%%%


%%%%%%%%%%%%%%%%%%%%%%%
%% Define the Solution Environment %%
%%%%%%%%%%%%%%%%%%%%%%%
\declaretheoremstyle
[
spaceabove=0pt, 
spacebelow=0pt, 
headfont=\normalfont\bfseries,
notefont=\mdseries, 
notebraces={(}{)}, 
headpunct={:\quad}, 
headindent={},
postheadspace={ }, 
postheadspace=4pt, 
bodyfont=\normalfont, 
qed=,
]{mystyle}

\declaretheorem[style=mystyle,title=解,numbered=no]{solution}
%%%%%%%%%%%%%%%%%%%%%%%
%% End of the Solution environment %%
%%%%%%%%%%%%%%%%%%%%%%%


%%%%%%%%%%%%%%%%%%%%
%% Put your information here %%
%%%%%%%%%%%%%%%%%%%
\newcommand{\name}{\text{胡睿}}  	          			%%% FILL IN YOUR NAME HERE
\newcommand{\id}{\text{PB17061124}}		        		%%% FILL IN YOUR ID HERE
\newcommand{\hwno}{七}                               %%%FILL IN homework number
%%%%%%%%%%%%%%%%%%%%
%% End of the student's info %%
%%%%%%%%%%%%%%%%%%%


\lhead{
	\textbf{\name}
}
\rhead{
	\textbf{\id}
}
\chead{\textbf{
		  作业 \hwno 
}}


\begin{document}
\vspace*{-4\baselineskip}
\thispagestyle{empty}

\begin{center}
{\bf\large 微机原理与嵌入式系统}\\
{2020春}\\
中国科学技术大学
\end{center}

\noindent
作业 \hwno  
\hfill
日期:{\today}
\\
姓名: \name             			
\hfill
学号: \id						
\hfill

\noindent
\rule{\textwidth}{2pt}

\medskip





%%%%%%%%%%%%%%%%%%%%%%%%%%%%%%%%%%%%%%%%%%%%%%%%%%%%%%%%%%%%%%%%
%% BODY OF HOMEWORK GOES HERE
%%%%%%%%%%%%%%%%%%%%%%%%%%%%%%%%%%%%%%%%%%%%%%%%%%%%%%%%%%%%%%%%

%\textbf{Notice, }to get the full credits, please show your solutions step by step.

\begin{exercise}[7.3]
	编写一个完整ARM汇编程序实现如下功能:当R3>R2时,将R2+10存入R3,否则将R2+100存入R3。
\end{exercise}
\begin{solution}
	汇编源代码:
	\begin{lstlisting}
	 ;AREA RESET, CODE, READONLY
	 ENTRY
	 MOV R2, #76;初始化R2的值
	 MOV R3, #88;初始化R3的值
	 CMP R3, R2;判断R3>R2?
	 ADDHI R3, R2, #10;R3>R2时,R3=R2+10
	 ADDLS R3, R2, #100;R3<=R2时,R3=R2+100
	 ;END
	\end{lstlisting}	
\end{solution}
\newpage

\begin{exercise}[7.4]
	将数据段中10个数据中的偶数个数统计后放入R0寄存器。	
\end{exercise}
\begin{solution}
	汇编源代码:
	\begin{lstlisting}

	
	\end{lstlisting}
\end{solution}
\newpage

\begin{exercise}[7.5]
	将数据段中10个有符号数中的正数个数统计后放入R0寄存器。	
\end{exercise}
\begin{solution}
	汇编源代码:
	\begin{lstlisting}
	
	
	\end{lstlisting}
\end{solution}
\newpage

\begin{exercise}[7.6]
	试编写一个循环程序,实现1至100的累加。	
\end{exercise}
\begin{solution}
	汇编源代码:
	\begin{lstlisting}
	
	
	\end{lstlisting}
\end{solution}
\newpage

\begin{exercise}[7.7]
	汇编程序如何定义子程序?如何调用子程序?	
\end{exercise}
\begin{solution}
	汇编源代码:
	\begin{lstlisting}
	
	
	\end{lstlisting}
\end{solution}
\newpage

\begin{exercise}[7.8]
	编写完整程序并利用汇编子程序计算N!(N<=10)。	
\end{exercise}
\begin{solution}
	汇编源代码:
	\begin{lstlisting}
	
	
	\end{lstlisting}
\end{solution}
\newpage

\begin{exercise}[7.9]
	编写完整汇编程序调用C函数计算N!(N<=10)。	
\end{exercise}
\begin{solution}
	汇编源代码:
	\begin{lstlisting}
	
	
	\end{lstlisting}
\end{solution}
\newpage

\begin{exercise}[7.10]
	C程序调用汇编函数计算字符串长度,并返回长度值。	
\end{exercise}
\begin{solution}
	汇编源代码:
	\begin{lstlisting}
	
	
	\end{lstlisting}
\end{solution}
\newpage

\begin{exercise}[7.11]
	阅读程序段,说明完成的功能。\\
	LOOP\\
	LDMIA R12!,(R0-R11)\\
	STMIA R13!, (R0-R11)\\
	CMP R12, R13\\
	BLO LOOP\\
\end{exercise}
\begin{solution}
	汇编源代码:
	\begin{lstlisting}
	
	
	\end{lstlisting}
\end{solution}
\newpage

\begin{exercise}[7.12]
	C程式中嵌入汇编有哪两种方式?Cortex M4可以使用哪种方式,为什么?
\end{exercise}
\begin{solution}
	汇编源代码:
	\begin{lstlisting}
	
	
	\end{lstlisting}
\end{solution}



%%%%%%%%%%%%%%%%%%%%%%%%%%%%%%%%%%%%%%%%%%%%%%%%%%%%%%%%%%%%%%%%

\end{document}
