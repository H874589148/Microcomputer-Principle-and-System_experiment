%%%%%%%%%%%%%%%%%%%%%%%%%%%%%%%%%%%%%%%%%%%%%%%%%%%%%%%%%%%%%%%%
%
%  Template for homework of DSP@ustc.
%
%  Fill in your name, student id, homework Number
%  
%   Please compile with XelaTex
%%%%%%%%%%%%%%%%%%%%%%%%%%%%%%%%%%%%%%%%%%%%%%%%%%%%%%%%%%%%%%%%


\documentclass[11pt,letter,notitlepage,UTF8]{ctexart}
%Mise en page
\usepackage{listings}
\usepackage{xcolor}
\lstset{
	numbers=left, 
	numberstyle= \tiny, 
	keywordstyle= \color{ blue!70},
	commentstyle= \color{red!50!green!50!blue!50}, 
	frame=shadowbox, % 阴影效果
	rulesepcolor= \color{ red!20!green!20!blue!20} ,
	escapeinside=``, % 英文分号中可写入中文
	xleftmargin=2em,xrightmargin=2em, aboveskip=1em,
	framexleftmargin=2em
}

\usepackage[left=2.5cm, right=2.5cm, lines=45, top=1.5in, bottom=0.7in]{geometry}
\usepackage{fancyhdr}
\usepackage{fancybox}
\usepackage{graphicx}
\usepackage{pdfpages} 
\usepackage{enumitem}
\usepackage{algorithm}
\usepackage{algorithmic}
\newcommand\Loadedframemethod{TikZ}
\usepackage[framemethod=\Loadedframemethod]{mdframed}

\usepackage{amssymb,amsmath}
\usepackage{amsthm}
\usepackage{thmtools}
\newtheorem{lemma}{Lemma}

\usepackage{subfigure}
\usepackage{threeparttable}

\pagestyle{fancy}
%%%%%%%%%%%%%%%%%%%%%%%%
%% Define the Exercise environment %%
%%%%%%%%%%%%%%%%%%%%%%%%
\mdtheorem[
topline=false,
rightline=false,
leftline=false,
bottomline=false,
leftmargin=-10,
rightmargin=-10
]{exercise}{\textbf{习题}}
%%%%%%%%%%%%%%%%%%%%%%%
%% End of the Exercise environment %%
%%%%%%%%%%%%%%%%%%%%%%%


%%%%%%%%%%%%%%%%%%%%%%%
%% Define the Solution Environment %%
%%%%%%%%%%%%%%%%%%%%%%%
\declaretheoremstyle
[
spaceabove=0pt, 
spacebelow=0pt, 
headfont=\normalfont\bfseries,
notefont=\mdseries, 
notebraces={(}{)}, 
headpunct={:\quad}, 
headindent={},
postheadspace={ }, 
postheadspace=4pt, 
bodyfont=\normalfont, 
qed=,
]{mystyle}

\declaretheorem[style=mystyle,title=解,numbered=no]{solution}
%%%%%%%%%%%%%%%%%%%%%%%
%% End of the Solution environment %%
%%%%%%%%%%%%%%%%%%%%%%%


%%%%%%%%%%%%%%%%%%%%
%% Put your information here %%
%%%%%%%%%%%%%%%%%%%
\newcommand{\name}{\text{胡睿}}  	          			%%% FILL IN YOUR NAME HERE
\newcommand{\id}{\text{PB17061124}}		        		%%% FILL IN YOUR ID HERE
\newcommand{\hwno}{七}                               %%%FILL IN homework number
%%%%%%%%%%%%%%%%%%%%
%% End of the student's info %%
%%%%%%%%%%%%%%%%%%%


\lhead{
	\textbf{\name}
}
\rhead{
	\textbf{\id}
}
\chead{\textbf{
		  作业 \hwno 
}}


\begin{document}
\vspace*{-4\baselineskip}
\thispagestyle{empty}

\begin{center}
{\bf\large 微机原理与嵌入式系统}\\
{2020春}\\
中国科学技术大学
\end{center}

\noindent
作业 \hwno  
\hfill
日期:{\today}
\\
姓名: \name             			
\hfill
学号: \id						
\hfill

\noindent
\rule{\textwidth}{2pt}

\medskip





%%%%%%%%%%%%%%%%%%%%%%%%%%%%%%%%%%%%%%%%%%%%%%%%%%%%%%%%%%%%%%%%
%% BODY OF HOMEWORK GOES HERE
%%%%%%%%%%%%%%%%%%%%%%%%%%%%%%%%%%%%%%%%%%%%%%%%%%%%%%%%%%%%%%%%

%\textbf{Notice, }to get the full credits, please show your solutions step by step.

\begin{exercise}[7.3]
	编写一个完整ARM汇编程序实现如下功能:当R3>R2时,将R2+10存入R3,否则将R2+100存入R3。
\end{exercise}
\begin{solution}
汇编源代码:
\begin{lstlisting}
	;AREA RESET, CODE, READONLY
	ENTRY
	MOV R2, #76;初始化R2的值
	MOV R3, #88;初始化R3的值
	CMP R3, R2;判断R3>R2?
	ADDHI R3, R2, #10;R3>R2时,R3=R2+10
	ADDLS R3, R2, #100;R3<=R2时,R3=R2+100
	;END
\end{lstlisting}
\end{solution}

\begin{exercise}[7.4]
	将数据段中10个数据中的偶数个数统计后放入R0寄存器。	
\end{exercise}
\begin{solution}
汇编源代码:
\begin{lstlisting}
	;AREA BUF, DATA, READWRITE;定义数据段Buf
Array DCB 0x11,0x22,0x33,0x44
	DCB 0x55,0x66,0x77,0x88
	DCB 0x00,0x00,0x00,0x00;定义12个字节的数组Array

	;AREA RESET, CODE, READONLY
	ENTRY
	LDR R0,=Array;取得数组Array的首地址
	MOV R1,#0;R1为循环计数器
	MOV R2,#0;R2为偶数计数器
LOOP CMP R1,#10;判断R1<10?
	BCS FOR_E;若条件失败(R1>=10)则退出循环
	LDR R3,[R0];从数组读取字到R3
	ANDS R3,R3,#0x01;取出最低位
	CMP R3,#1;判断R3>1?
	ADDNE R2,R2,#1;如果R3不等于1则R2++
	ADDS R1,R1,#1;低32位相加,影响标志位,保存进位,结果放入循环计数器R1
	ADDS R0,R0,#1;低32位相加,影响标志位,保存进位,结果放入数组地址R0
	B LOOP
FOR_E MOV R0, R2;退出循环R0=R2
	END
	
\end{lstlisting}
\end{solution}

\begin{exercise}[7.5]
	将数据段中10个有符号数中的正数个数统计后放入R0寄存器。	
\end{exercise}
\begin{solution}
汇编源代码:
\begin{lstlisting}
	;AREA BUF, DATA, READWRITE;定义数据段Buf
Array DCB 0x11,0x22,0x33,0x44
	DCB 0x55,0x66,0x77,0x88
	DCB 0x00,0x00,0x00,0x00;定义12个字节的数组Array

	;AREA RESET, CODE, READONLY
	ENTRY
	LDR R0,=Array;取得数组Array的首地址
	MOV R1,#0;R1为循环计数器
	MOV R2,#0;R2为偶数计数器
LOOP CMP R1,#10;判断R1<10?
	BCS FOR_E;若条件失败(R2>=10)则退出循环
	LDR R3,[R0];从数组读取字到R3
	ANDS R3,R3,#0x80;取出符号位
	CMP R3,#1;判断R3>1?
	ADDNE R2,R2,#1;如果R3不等于1则R2++
	ADDS R1,R1,#1;低32位相加,影响标志位,保存进位,结果放入循环计数器R1
	ADDS R0,R0,#1;低32位相加,影响标志位,保存进位,结果放入数组地址R0
	B LOOP
FOR_E MOV R0, R2;退出循环R0=R2
	END

\end{lstlisting}
\end{solution}

\begin{exercise}[7.6]
	试编写一个循环程序,实现1至100的累加。	
\end{exercise}
\begin{solution}
汇编源代码:
\begin{lstlisting}
	;AREA RESET, CODE, READONLY
	EXPORT RESULT [WEAK];声明一个可以全局引用的标号RESULT
RESULT;输出结果
	ENTRY
	MOV R0, #100;循环数,即累加个数100
	MOV R1, #0;初始化数据
LOOP ADD R1, R1, R0;将数据进行相加,获得最后的数据
	SUBS R0, R0, #1;R0=R0-1
	CMP R0, #0;将R0与0比较判断循环是否结束
	BNE LOOP;判断循环是否结束,结束则进行下面的步骤
	LDR R2, =RESULT;RESULT为数据段存储结果单元,将RESULT地址赋给R2
	STR R1, [R2]
	;END

\end{lstlisting}
\end{solution}

\begin{exercise}[7.7]
	汇编程序如何定义子程序?如何调用子程序?	
\end{exercise}
\begin{solution}
汇编源代码:
\begin{lstlisting}
X EQU 19;定义X的值为19
N EQU 20;定义N的值为20
	;AREA RESET, CODE, READONLY;声明代码段
	ENTRY;标识程序入口
START LDR R0, =X ;给R0、R1赋初值
	LDR R1 , =N;将N读取到R1
	BL MAX;调用子程序MAX
HALT B HALT;
MAX;声明子程序MAX
	CMP R0, R1;判断R0>R1?
	MOVHI R2, R0;R0>R1时R2=R0,R2等于R0,R1中的最大值
	MOVLS R2, R1;R0<=R1时R2=R1,R2等于R0,R1中的最大值
	MOV PC, LR;返回语句
	;END

\end{lstlisting}
\end{solution}

\begin{exercise}[7.8]
	编写完整程序并利用汇编子程序计算N!(N<=10)。	
\end{exercise}
\begin{solution}
汇编源代码:
\begin{lstlisting}
N EQU 10;定义N的值为10
	;AREA RESET, CODE, READONLY;声明代码段
	ENTRY;标识程序入口
START LDR R0, =N;给R0、R1赋初值R0=10,R1=1
	MOV R1, #1;初始化R1
	BL FUNC;调用子程序MAX
	MOV R2, R1;R2=R1
HALT B HALT;
FUNC;声明子程程序FUNC
LOOP MUL R2, R2, R1;R2=R2*R1
	ADD R1, R1, #1;R1++
	CMP R1, R0;将R1与R0比较看循环是否结束
	BNE LOOP;判断循环是否结束,结束则进行下面的步骤
	MOV PC, LR;返回语句
	;END

\end{lstlisting}
\end{solution}

\begin{exercise}[7.9]
	编写完整汇编程序调用C函数计算N!(N<=10)。	
\end{exercise}
\begin{solution}
汇编源代码:
\begin{lstlisting}
;N EQU 10;定义N的值为10
;AREA RESET, CODE, READONLY;声明代码段
ENTRY;标识程序入口
IMPORT FACT;
MOV R0, #10;初始化R0
BL FACT
LDR R1,=0x20000000;结果R0存入内存单元R1
STR R0,[R1]

;END

\end{lstlisting}
C语言源代码:
\begin{lstlisting}
//hr0709.c
//需要调用的C函数原型计算N!
int FACT(int n)
{
	int i=1;
	int result=1;
	if(n==0)
		result=1;
	else if(n<0)
		result=0;
	else{
		for(i=1;i<=n;i++){
			result=result*i;
		}
	}
	return(result);
}

\end{lstlisting}
\end{solution}

\begin{exercise}[7.10]
	C程序调用汇编函数计算字符串长度,并返回长度值。	
\end{exercise}
\begin{solution}
C语言源代码:
\begin{lstlisting}
//hr0710.c
extern void my_strlen(char * d) ; //需要调用的汇编函数原型并加extern关键字
int main ()
{
	char * srcstr = "0123456" ;
	my_strlen(srcstr);
	return 0 ;
}
	
\end{lstlisting}
汇编源代码:
\begin{lstlisting}
	;N EQU 10;定义N的值为10
	;AREA RESET, CODE, READONLY;声明代码段
	ENTRY;标识程序入口
	EXPORT my_strlen;
my_strlen
	MOV R1, #0;初始化R3
LOOP LDRB R2, [R0], #1;R0指向源字符串地址,取出字符内容存入R2
	ADD R1, R1, #1;R1=R1+1
	CMP R2, #0;判断R2是否为0
	BNE LOOP;判断循环是否结束,结束则进行下面的步骤
	BLX LR;返回指令
	;END

\end{lstlisting}
\end{solution}

\begin{exercise}[7.11]
	阅读程序段,说明完成的功能。\\
	LOOP\\
	LDMIA R12!,(R0-R11)\\
	STMIA R13!, (R0-R11)\\
	CMP R12, R13\\
	BLO LOOP\\
\end{exercise}




%%%%%%%%%%%%%%%%%%%%%%%%%%%%%%%%%%%%%%%%%%%%%%%%%%%%%%%%%%%%%%%%

\end{document}
